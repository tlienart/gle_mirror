%%%%%%%%%%%%%%%%%%%%%%%%%%%%%%%%%%%%%%%%%%%%%%%%%%%%%%%%%%%%%%%%%%%%%%%%
%                                                                      %
% GLE - Graphics Layout Engine <http://www.gle-graphics.org/>          %
%                                                                      %
% Modified BSD License                                                 %
%                                                                      %
% Copyright (C) 2009 GLE.                                              %
%                                                                      %
% Redistribution and use in source and binary forms, with or without   %
% modification, are permitted provided that the following conditions   %
% are met:                                                             %
%                                                                      %
%    1. Redistributions of source code must retain the above copyright %
% notice, this list of conditions and the following disclaimer.        %
%                                                                      %
%    2. Redistributions in binary form must reproduce the above        %
% copyright notice, this list of conditions and the following          %
% disclaimer in the documentation and/or other materials provided with %
% the distribution.                                                    %
%                                                                      %
%    3. The name of the author may not be used to endorse or promote   %
% products derived from this software without specific prior written   %
% permission.                                                          %
%                                                                      %
% THIS SOFTWARE IS PROVIDED BY THE AUTHOR "AS IS" AND ANY EXPRESS OR   %
% IMPLIED WARRANTIES, INCLUDING, BUT NOT LIMITED TO, THE IMPLIED       %
% WARRANTIES OF MERCHANTABILITY AND FITNESS FOR A PARTICULAR PURPOSE   %
% ARE DISCLAIMED. IN NO EVENT SHALL THE AUTHOR BE LIABLE FOR ANY       %
% DIRECT, INDIRECT, INCIDENTAL, SPECIAL, EXEMPLARY, OR CONSEQUENTIAL   %
% DAMAGES (INCLUDING, BUT NOT LIMITED TO, PROCUREMENT OF SUBSTITUTE    %
% GOODS OR SERVICES; LOSS OF USE, DATA, OR PROFITS; OR BUSINESS        %
% INTERRUPTION) HOWEVER CAUSED AND ON ANY THEORY OF LIABILITY, WHETHER %
% IN CONTRACT, STRICT LIABILITY, OR TORT (INCLUDING NEGLIGENCE OR      %
% OTHERWISE) ARISING IN ANY WAY OUT OF THE USE OF THIS SOFTWARE, EVEN  %
% IF ADVISED OF THE POSSIBILITY OF SUCH DAMAGE.                        %
%                                                                      %
%%%%%%%%%%%%%%%%%%%%%%%%%%%%%%%%%%%%%%%%%%%%%%%%%%%%%%%%%%%%%%%%%%%%%%%%

\chapter{Programming Facilities}

\section{Expressions}
\index{variables}
\index{expressions}
Wherever GLE is expecting a number it can be replaced with an expression.  For
example

\preglecode{}
\begin{Verbatim}
     rline 3 2
\end{Verbatim}
\postglecode{}

and

\preglecode{}
\begin{Verbatim}
     rline 9/3 sqrt(4)
\end{Verbatim}
\postglecode{}

will produce the same result.

An expression in GLE is delimited by white space, so it may not contain any
spaces - `{\sf rline 3*3 2}' is valid but `{\sf rline 3 * 3 2}' will not work.

Or `{\sf let d2 = 3+sin(d1)}' will work and `{\sf let d2= 3 + sin(d1)}' won't.

Expressions may contain numbers, arithmetic operators ({\sf +}, {\sf -}, {\sf *},
{\sf /}, \verb+^+ (to the power of)), relational operators ($>$, $<$, $=>$, $<=$,
$=$, $<>$) Boolean operators ({\sf and}, {\sf or}), variables and built-in functions.

When GLE is expecting a colour or marker name (like `green' or `circle')
it can be given a string variable, or an expression enclosed in braces.

\section{Functions Inside Expressions}

\begin{figure}[tb]
\centering
\mbox{\includegraphics{prog/fig/axisformat}}
\caption{\label{axformat:fig}Examples of number formatting options.}
\end{figure}

\begin{commanddescription}

\item[{\sf eval(str)}]
\index{eval()}
Evaluates the given string as if it was a GLE expression and returns the result. E.g., \texttt{eval("3+4")} returns 7.

\item[{\sf arg(i), arg\$(i), nargs()}]
\label{args:cmd}
\index{arg()}
\index{arg\$()}
\index{nargs()}
Provide access to the command line arguments that are passed to GLE. This is useful for generating multiple similar plots from a single script. arg(i) returns the i-the argument (as a number), arg\$(i) returns the i-the argument as a string, and nargs() returns the number of arguments. Only arguments that come after the name of the GLE script are counted. For example, if GLE is run as:

\begin{Verbatim}
     gle -o graph-1.eps graph.gle "Title" 0.5
\end{Verbatim}

\noindent{}then nargs() returns 2, arg\$(1) returns ``Title'', and arg(2) returns 0.5.

The typical use of these functions is to create a script ``graph.gle'' as follows:

\begin{Verbatim}
size 10 10
begin graph
   title arg$(1)
   data arg$(2)
   d1 line color red
end graph
\end{Verbatim}

\noindent{}and then creating different graphs by running GLE multiple times:

\begin{Verbatim}
     gle -o beans.eps graph.gle "Beans" "beans.csv"
     gle -o peas.eps graph.gle "Peas" "peas.csv"
\end{Verbatim}

\noindent{}This will create two graphs: ``beans.eps'' and ``peas.eps''. The arg() functions can be used at all places in the script where an expression is expected. They can even be used in place of GLE commands in a graph block by means of the $\backslash{}$expr() function. For example,

\begin{Verbatim}
     data "file.csv"
     d\expr{arg(1)} line color red
\end{Verbatim}

\noindent{}in the graph block will allow one to draw different datasets from a single file on multiple graphs. To do so, run:

\begin{Verbatim}
     gle -o d1.eps graph.gle 1
     gle -o d2.eps graph.gle 2
\end{Verbatim}

\item[{\sf dataxvalue({\it ds},{\it i}), datayvalue({\it ds},{\it i}), ndata({\it ds})}]
\index{dataxvalue}\label{dataxvalue}
\index{datayvalue}\label{datayvalue}
\index{ndata}\label{ndata}

These functions can be used to retrieve the data values from a given dataset. They only work after data has been loaded by means of a graph block (Ch.~\ref{graph:chap}).

A dataset is specified with a dataset identifier {\it ds} (a string of the form ``d$i$'', with $i$ an integer). The function `{\sf ndata}' returns the number of points in the dataset, and the functions `{\sf dataxvalue}' and `{\sf datayvalue}' return the $x$ and $y$ values of point {\it i}, where {\it i} ranges from 1 to {\sf ndata({\it ds})}.

The following example shows how these functions can be used to compute the maximum of a dataset:

\begin{Verbatim}
sub dmaxy ds$
   local crmax = datayvalue(ds$,1)
   for i = 2 to ndata(ds$)
      crmax = max(crmax, datayvalue(ds$,i))
   next i
   return crmax
end sub
\end{Verbatim}

This subroutine together with other subroutines for computing the minimum, mean, area, etc. of a dataset are defined in the include file `graphutil.gle', which is distributed together with GLE.

\item[{\sf format\$({\it exp},{\it format})}]
\index{format\$()} \label{formatnum:pg}

Returns a string representation of {\it exp} formatted as specified in {\it format}.

Basic formats:
\begin{itemize}
\item {\sf append {\it s}}: appends the string $s$ after the formatted number. This can be used to add a unit.

\item {\sf dec}, {\sf hex [upper|lower]}, {\sf bin}: format as decimal, hexadecimal (upper-case or lower-case), or binary.

\item {\sf fix} {\it places}: format with {\it places} decimal places.

\item {\sf percent} {\it places}: format as ({\it exp} $\cdot 100$)\% with {\it places} decimal places.

\item {\sf sci} {\it sig} {\sf [e,E,10]} {\sf [expdigits {\it num}]} {\sf [expsign]}: format in scientific notation with {\it sig} significant digits. Use `e', `E', or `10' as notation for the exponent. With the option {\sf expdigits} the number of digits in the exponent can be set and {\sf expsign} forces a sign in the exponent.

\item {\sf eng} {\it digits} {\sf [e,E,10]} {\sf [expdigits {\it num}]} {\sf [expsign]} {\sf [num]}: format in engineering notation. The options are similar to `sci'. If the option `num' is given, then numeric notation is used instead of, e.g., $\mu$, m, k, M.

\item {\sf round} {\it sig}: format a number with {\it sig} significant digits.

\item {\sf frac}: format the number as a fraction.

\item {\sf pi}: format the number as a fraction times $\pi$ (E.g., xaxis labels of Fig.~\ref{grscale:fig}).
\end{itemize}

Options for all formats:
\begin{itemize}
\item {\sf nozeroes}: remove unnecessary zeroes at the end of the number.

\item {\sf sign}: also include a sign for positive numbers.

\item {\sf pad {\it nb} [left] [right]}: pad the result with spaces from the left or right.

\item {\sf prefix {\it nb}}: prefix the number with zeroes so that {\it nb} digits are obtained.

\item {\sf prepend {\it s}}: prepends the string $s$ before the formatted number.

\item {\sf min {\it val}}: use format for numbers $\ge$ {\it val}.

\item {\sf max {\it val}}: use format for numbers $\le$ {\it val}.
\end{itemize}

Examples:
\begin{itemize}
\item format\$(3.1415, ``fix 2'') = 3.14
\item format\$(3756, ``round 2'') = 3800
\item format\$(3756, ``sci 2 10 expdigits 2'') = $3.8\cdot10^{03}$
\end{itemize}

Several formats can be combined into one string: "sci 2 10 min 1e2 fix 0" uses scientific notations for numbers above $10^2$ and decimal notation for smaller numbers. See Fig.~\ref{axformat:fig} for more examples.

\item[{\sf pagewidth(), pageheight()}]
\index{pagewidth()} \index{pageheight()}
These functions return the width and height of the output. These are the values set with the `\texttt{size}' command.

\item[{\sf pointx(), pointy()}]\label{fct:pointxy}
These functions return the x and y values of a named point.

\preglecode{}
\begin{Verbatim}
     begin box add 0.1 name mybox
         write "Hello"
     end box
     amove pointx(mybox.bc) pointy(mybox.bc)
     rline 0 -2 arrow end
\end{Verbatim}
\postglecode{}

\item[{\sf twidth(str), theight(str), tdepth(str)}]
\index{twidth()} \index{theight()}
These functions return the width, depth and height of a string, if it was
printed in the current font and size.

\item[{\sf width(obj), height(obj)}]\label{fct:wdhi}
\index{width()} \index{height()}
These functions return the width and height of a named object.

\item[{\sf xend(), yend()}]
\index{xend()} \index{yend()}
These functions return the end point of the last thing drawn. This is
of particular interest when drawing text.

\preglecode{}
\begin{Verbatim}
     text abc
     set color blue
     text def
\end{Verbatim}
\postglecode{}

This would draw the {\sf def} on top of the {\sf abc}.  To draw the {\sf def}
immediately following the {\sf abc} simply do the following (Note that
absolute move is used, not relative move):

\begin{minipage}[c]{8cm}
\begin{Verbatim}
set just left
text abc
set color gray20 
amove xend() yend() 
text def            
\end{Verbatim}
\end{minipage}
\hfill
\begin{minipage}[c]{7cm}
\mbox{\includegraphics{primitives/fig/gc_xend}}
\end{minipage}

\item[{\sf xg(), yg()}]
\index{xg()} \index{yg()}
\index{functions}
With these functions it is possible to move to a position on a graph
using the graph's axis units.
To draw a filled box on a graph, at position x=948, y=.004  measured
on the graph axis:

\preglecode{}
\begin{Verbatim}
     begin graph
        xaxis min 100 max 2000
        yaxis min -.01 max .01
        ...
     end graph
     amove xg(948) yg(.004)
     box 2 2 fill gray10
\end{Verbatim}
\postglecode{}

\item[{\sf xpos(), ypos()}]
\index{xpos()} \index{ypos()}
Returns the current x and y points.
\end{commanddescription}

\noindent{}See Appendix~\ref{fct:sec} for an overview of all functions provided by GLE.

\section{Using Variables}
\index{variables}
\index{string!variables}

GLE has two types of variables,  floating point and string. String 
variables always end with a dollar sign.  A string variable contains
text like ``Hello, this is text'', a floating point variable can
only store numbers like 1234.234.

\preglecode{}
\begin{Verbatim}
name$ = "Joe"
height = 6.5   ! Height of person
shoe = 0.05    ! shoe adds to height of person
amove 1 1
box 0.2 height+shoe
write name$
\end{Verbatim}
\postglecode{}

\section{String constants}
\index{string!constants}

String constants can be double quoted or single quoted. To include a double quote character in a double quoted string it should be doubled. The same holds for a single quoted string. Backslash characters are not interpreted in a special way by GLE's parser. (They are, however, interpreted by the built-in \TeX{} system used in the `{\sf write}' command.) Here are some examples:

\preglecode{}
\begin{Verbatim}
print "Double quoted string"
print 'Single quoted string'
print "Between ""double quotes"""
print "{\it hello}"
print "Three double quotes """""""
print """"
\end{Verbatim}
\postglecode{}

The result of these print commands is:

\preglecode{}
\begin{Verbatim}
Double quoted string
Single quoted string
Between "double quotes"
{\it hello}
Three double quotes """
"
\end{Verbatim}
\postglecode{}

\section{Programming Loops}
\index{loops}
\index{for}
\index{next}

The simple way to draw a 6 $\times$ 8 grid would be to use a whole mass 
of line commands:

\preglecode{}
\begin{Verbatim}
amove 0 0 
rline 0 8 
amove 1 0 
rline 1 8 
...
amove 6 0 
rline 6 8
\end{Verbatim}
\postglecode{}

this would be laborious to type in, and would become 
impossible to manage with several grids.  By using a simple loop
this can be avoided:

\preglecode{}
\begin{Verbatim}
for x = 0 to 6
   amove x 0 
   rline x 8 
next x
for y = 0 to 8
   amove 0 y 
   rline 6 y 
next y 
\end{Verbatim}
\postglecode{}

For-next loops, and all other control constructs, can also be used among others inside graph and key blocks. This is useful for drawing many similar functions (Fig.~\ref{fig:sqroot}). Besides for-next loops, GLE also supports while and until loops:

\index{while}
\preglecode{}
\begin{Verbatim}
i = 0
while i <= 10
   print "Value: " i
   i = i + 1
next
\end{Verbatim}
\postglecode{}

\index{until}
\preglecode{}
\begin{Verbatim}
i = 0
until i > 10
   print "Value: " i
   i = i + 1
next
\end{Verbatim}
\postglecode{}

\section{If-then-else}
\index{if}
\index{then}
\index{else}

GLE supports if-then-else statements as follows:

\preglecode{}
\begin{Verbatim}
if a < 1 then print a "is smaller than 1"
else if a < 2 then print a "is smaller than 2 but larger than 1"
else if a < 3 then print a "is smaller than 3 but larger than 2"
else print a "is larger than 3"
\end{Verbatim}
\postglecode{}

\noindent{}to create blocks of code for the `then' and `else' branches, instead use:

\preglecode{}
\begin{Verbatim}
if a < 1 then
   print a "is smaller than 1"
   ...
else
   ...
end if
\end{Verbatim}
\postglecode{}

More complex conditions can be created with the logic connectives `{\sf and}', `{\sf or}', and `{\sf not}' (note the parenthesis around the logical expressions):

\preglecode{}
\begin{Verbatim}
if (a >= 1) and (a <= 10) then print "a is between 1 and 10"
\end{Verbatim}
\postglecode{}

\section{Subroutines}
\index{subroutines}

To draw lots of grids all of different dimensions a subroutine can
be defined and then used again and again:

\preglecode{}
\begin{Verbatim}
sub grid nx ny 
   local x, y
   begin origin
      for x = 0 to nx
         amove x 0 
         aline x ny 
      next x
      for y = 0 to ny
         amove 0 y 
         aline nx y 
      next y 
   end origin
end sub
     
amove 2 4
grid  6 8
amove 2 2
grid  9 5
\end{Verbatim}
\postglecode{}

Inside a subroutine, the keyword `\texttt{local}'\index{local} can be used to define local variables. E.g., `\texttt{local x = 3}', defines the local variable `x' and assigns it the value 3. It is also possible to define several local variables at once, as is shown in the `grid' example above.

The keyword `\texttt{return}'\index{return} can be used to return a value from a subroutine. E.g.,

\preglecode{}
\begin{Verbatim}
sub gaussian x mu sigma
   return 1/(sigma*sqrt(2*pi))*exp(-((x-mu)^2)/(2*sigma^2))
end sub
\end{Verbatim}
\postglecode{}

The main GLE file will be much easier to manage if subroutine definitions are moved into a separate file:

\preglecode{}
\begin{Verbatim}
include "griddef.gle"

amove 2 4
grid  2 4
amove 2 2
grid  9 5
\end{Verbatim}
\postglecode{}

More information about the ``include''\index{include} command can be found on page~\pageref{incl:cmnd}.

\subsection{Default Arguments}

Given the following subroutine definition:

\preglecode{}
\begin{Verbatim}
sub mysub x y color$ fill$
   default color  "black"
   default fill   "clear"
   print "Color: " color$
   print "Fill:  " fill$
end sub
\end{Verbatim}
\postglecode{}

\noindent{}the following calls are valid:

\preglecode{}
\begin{Verbatim}
mysub 1 0
mysub 1 0 red
mysub 1 0 red green
mysub 1 0 fill blue
mysub 1 0 color red
mysub 1 0 color red fill blue
\end{Verbatim}
\postglecode{}

\section{Forward Declarations}

A forward declaration of a subroutine is possible with:

\preglecode{}
\begin{Verbatim}
declare sub mysub x y
\end{Verbatim}
\postglecode{}

\noindent{}Forward declarations are required for declaring mutually recursive subroutines.

\section{I/O Functions}
\index{I/O-functions}
\index{fopen} \index{fread} \index{freadln}
\index{fclose} \index{fwrite} \index{fwriteln}
\index{fgetline} \index{ftokenizer}

The following I/O functions are available:

\begin{commanddescription}
\item[{\sf fopen {\it name} {\it file} [read$|$write]}]

Open the file ``{\it name}'' for reading or for writing. The resulting file handle is stored in variable ``{\it file}'' and must be passed to all other I/O functions.

\item[{\sf fclose {\it file}}]

Close the given file.

\item[{\sf fread {\it file} {\it x1} $\ldots$}]
\item[{\sf freadln {\it file} {\it x1} $\ldots$}]

Read entries from ``{\it file}'' into given arguments {\it x1} $\ldots$

\item[{\sf fwrite {\it file} {\it x1} $\ldots$}]

Write given arguments to ``{\it file}''.

\item[{\sf fwriteln {\it file} {\it x1} $\ldots$}]

Write given arguments to ``{\it file}'' and start a new line.

\item[{\sf fgetline {\it file} {\it line\$}}]

Read an entire line from ``{\it file}'' and store it in the string ``{\it line\$}''.

\item[{\sf ftokenizer {\it file} {\it commentchar spacetokens singletokens}}]

Sets up the parameters of the tokenizer that controls the reading of ``{\it file}''. The {\it commentchar} parameter specifies the characters that are to be interpreted as line comments. It is a string, but each character of the string is assumed to be a separate comment character. The default is ``!''. If one would write ``!\%'', then both ``!'' and ``\%'' would be comment indicators. The {\sf fread} functions skip everything after a comment character to the end of the line. The {\it spacetokens} string represents the set of characters that are interpreted as spaces or delimiters. The default value is `` ,$\backslash$t$\backslash$r$\backslash$n'', i.e., space, comma, tab, carriage return, and newline are delimiters by default. Finally, the {\it singletokens} string identifies characters that should be returned as separate tokens, even if they are glued to other tokens. For example, if ``@'' would be a single char token, then the string ``me@myself.com'' would be returned in three tokens: ``me'', ``@'', and ``myself.com''. 

\end{commanddescription}

For example:

\preglecode{}
\begin{Verbatim}
fopen "file.dat" f1 read
fopen "file.out" f2 write
until feof(f1)
        fread f1 x y z
        aline x y
        rline x z
        fwriteln f2 x*2 "y =" y
next
fclose f1
fclose f2
\end{Verbatim}
\postglecode{}

\section{Device Dependent Control}
\index{device control}
A built in function which returns a string describing 
the device is available. \\e.g. \verb# DEVICE$() = "HARDCOPY, PS,"#\\
on the postscript driver.

This can be used to use particular fonts etc on appropriate
devices. E.g.:

\preglecode{}
\begin{Verbatim}
                if pos(device$(),"PS,",1)>0 then
                        set font psncsb
                end if 
\end{Verbatim}
\postglecode{}
