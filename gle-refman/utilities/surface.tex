%%%%%%%%%%%%%%%%%%%%%%%%%%%%%%%%%%%%%%%%%%%%%%%%%%%%%%%%%%%%%%%%%%%%%%%%
%                                                                      %
% GLE - Graphics Layout Engine <http://www.gle-graphics.org/>          %
%                                                                      %
% Modified BSD License                                                 %
%                                                                      %
% Copyright (C) 2009 GLE.                                              %
%                                                                      %
% Redistribution and use in source and binary forms, with or without   %
% modification, are permitted provided that the following conditions   %
% are met:                                                             %
%                                                                      %
%    1. Redistributions of source code must retain the above copyright %
% notice, this list of conditions and the following disclaimer.        %
%                                                                      %
%    2. Redistributions in binary form must reproduce the above        %
% copyright notice, this list of conditions and the following          %
% disclaimer in the documentation and/or other materials provided with %
% the distribution.                                                    %
%                                                                      %
%    3. The name of the author may not be used to endorse or promote   %
% products derived from this software without specific prior written   %
% permission.                                                          %
%                                                                      %
% THIS SOFTWARE IS PROVIDED BY THE AUTHOR "AS IS" AND ANY EXPRESS OR   %
% IMPLIED WARRANTIES, INCLUDING, BUT NOT LIMITED TO, THE IMPLIED       %
% WARRANTIES OF MERCHANTABILITY AND FITNESS FOR A PARTICULAR PURPOSE   %
% ARE DISCLAIMED. IN NO EVENT SHALL THE AUTHOR BE LIABLE FOR ANY       %
% DIRECT, INDIRECT, INCIDENTAL, SPECIAL, EXEMPLARY, OR CONSEQUENTIAL   %
% DAMAGES (INCLUDING, BUT NOT LIMITED TO, PROCUREMENT OF SUBSTITUTE    %
% GOODS OR SERVICES; LOSS OF USE, DATA, OR PROFITS; OR BUSINESS        %
% INTERRUPTION) HOWEVER CAUSED AND ON ANY THEORY OF LIABILITY, WHETHER %
% IN CONTRACT, STRICT LIABILITY, OR TORT (INCLUDING NEGLIGENCE OR      %
% OTHERWISE) ARISING IN ANY WAY OUT OF THE USE OF THIS SOFTWARE, EVEN  %
% IF ADVISED OF THE POSSIBILITY OF SUCH DAMAGE.                        %
%                                                                      %
%%%%%%%%%%%%%%%%%%%%%%%%%%%%%%%%%%%%%%%%%%%%%%%%%%%%%%%%%%%%%%%%%%%%%%%%

\chapter{Surface and Contour Plots}
\label{surf:chap}

\section{Surface Primitives}

\subsection{Overview}

\index{Surface}
Surface plots three dimensional data using a wire frame with hidden line removal.

The simplist surface code would look like this.
\preglecode{}
\begin{Verbatim}
begin surface
   data "myfile.z" 5 5
end surface
\end{Verbatim}
\postglecode{}

The surface block can contain the following commands:\\
{\sf size {\it x }  {\it y }}\\	
{\sf cube [off]  [xlen {\it v}\ ] [ylen {\it v}\ ] [zlen {\it v}\ ] [nofront]  [lstyle {\it l\ }]  [color {\it c\ }]  }\\	
{\sf data {\it myfile.z}  [xsample {\it n1}\ ] [ysample {\it n2}\ ] [sample {\it n3}\ ]  [nx {\it n1}\ ] [ny {\it n2}\ ] }\\
{\sf harray {\it n} \ }\\
{\sf xlines ---  ylines [off]}\\
{\sf xaxis  --- yaxis --- zaxis [min {\it v}\ ]  [max {\it v}\ ]  [step {\it v}\ ]  [color {\it c}\ ] [lstyle {\it l}\ ] [hei {\it v}\ ] [off] }\\
{\sf xtitle ---	ytitle --- ztitle "{\sf title}"  [dist {\it v}\ ] [color {\it c}\ ] [hei {\it v}\ ] }\\
{\sf title "{\it main title}"  [dist {\it v}\ ] [color {\it c}\ ] [hei {\it v}\ ] }\\
{\sf rotate $\theta$\  $\phi$\  x}\\
{\sf view x y p}\\
{\sf top --- underneath \  [off]\ [lstyle {\it n} \ ]\ [color {\it c}\ ]}\\
{\sf back [zstep {\it v}\ ] [ystep {\it v}\ ] [lstyle {\it l}\ ] [color {\it c}\ ] [nohidden] }\\
{\sf base [xstep {\it v}\ ] [ystep {\it v}\ ] [lstyle {\it l}\ ] [color {\it c}\ ] [nohidden] }\\
{\sf right [zstep {\it v}\ ] [xstep {\it v}\ ] [lstyle {\it l}\ ] [color {\it c}\ ] [nohidden] }\\
{\sf skirt on}\\
{\sf points {\it myfile.dat} }\\
{\sf marker {\it circle}\  [hei {\it v} \ ] [color {\it c} \ ] }\\
{\sf droplines --- riselines \ [color {\it c}\ ] [lstyle {\it n}\ ]}\\
{\sf zclip [min {\it v1}\ ]  [max {\it v2}\ ]}\\
%
%

\subsection{Surface Commands}
\begin{commanddescription}
\item[{\sf size {\it x} \  {\it y}}]
\index{size}
Specifies the size in cm to draw the surface.  The 3d cube
will fit inside this box.  The default is 18cm x 18cm 
e.g. \verb# size 10 10#

\item[{\sf cube [off]  [xlen {\it v}\ ] [ylen {\it v}\ ] [zlen {\it v}\ ] [nofront]  [lstyle {\it l}\ ]  [COLOR {\it c}\ ]  }]
Surface is drawing a 3d cube.

\begin{tabular}{ll}
off     & Stops GLE from drawing the cube.\\
xlen    & The length of the cubes x dimension in cm.\\
nofront & Removes the front three lines of the cube.\\
lstyle  & Sets the line style to use drawing the cube.\\
color   & Sets the color of lines to use drawing the cube.\\
\end{tabular}

\begin{minipage}[c]{8cm}
\begin{Verbatim}
begin surface
   size 7 7
   data "jack.z"
   cube zlen 13
   top color orange
   underneath color red
end surface
\end{Verbatim}
\end{minipage}
\hfill
\begin{minipage}[c]{7cm}
\mbox{\includegraphics{utilities/fig/surf5}}
\end{minipage}

\item[{\sf data {\it myfile.z}  [xsample {\it n1}\ ] [ysample {\it n2}\ ] [sample {\it n3}\ ]  [nx {\it n1}\ ] [ny {\it n2}\ ]}]
\index{data}\label{zfile:pg}
Loads a file of Z values in.  The NX and NY dimensions
are optional, normally the dimensions of the data will
be defined on the first line of the data file. e.g.
\begin{verbatim}
	! nx 10  ny 20    xmin 1 xmax 10 ymin 1 ymax 20
	1 2 42 4 5 2 31 4 3 2 4
	1 2 42 4 5 2 31 4 3 2 4 etc...
	
	y1,x1, y1,x2	...	y1,xn
	y2,x1, y2,x2	...	y2,xn
        .
	.
	.
	yn,x1, yn,x2	...	yn,xn
\end{verbatim}

Data files can be created using LETZ or FITZ.
LETZ will create a data file from an x,y function.
FITZ will create a data file from a list of x,y,z data points.

\begin{tabular}{ll}
xsample & Tells surface to only read every n'th data point from\\
        & the data file.  This speeds things up while you are\\
        & messing around.\\[1.5ex]
ysample & Tells surface to only read every n'th line from the\\
        & data file.\\[1.5ex]
sample  & Sets both xsample and ysample\\
\end{tabular}
(see also POINTS)

\begin{minipage}[c]{8cm}
\begin{Verbatim}
begin surface
   size 5 5 
   xtitle "X-axis"
   ytitle "Y-axis"
   data   "surf1.z"
end surface
\end{Verbatim}
\end{minipage}
\hfill
\begin{minipage}[c]{7cm}
\mbox{\includegraphics{utilities/fig/surf1}}
\end{minipage}

\item[{\sf harray {\it n}}]
\index{harray}
The hidden line removal is accomplished with the help of
an array of heights which record the current horizon,  the
quality of the output is proportional to the width of this
array.  (also the speed of output)

To get good quality you may want to increase this from
the default of about 900 to 2 or 3 thousand. 
e.g. \verb# harray 2000#

\item[{\sf xlines off}]
\index{xlines}
Stops SURF from drawing lines of constant X.

\item[{\sf ylines off}]
\index{ylines}
Stops SURF from drawing lines of constant Y.

\item[{\sf xaxis [min {\it v}\ ]  [max {\it v}\ ]  [step {\it v}\ ]  [color {\it
c}\ ] [lstyle {\it l}\ ] [hei {\it v}\ ] [off] }]
\index{xaxis}

\item[{\sf zaxis [min {\it v}\ ]  [max {\it v}\ ]  [step {\it v}\ ]  [color {\it
c}\ ] [lstyle {\it l}\ ] [hei {\it v}\ ] [off] }]
\index{zaxis}

\item[{\sf yaxis [min {\it v}\ ]  [max {\it v}\ ]  [step {\it v}\ ]  [color {\it
c}\ ] [lstyle {\it l}\ ] [hei {\it v}\ ] [off] }]
\index{yaxis}

\begin{tabular}{ll}
min,max & Set the range used for labelling the axis.\\
step    & The distance between labels on the axis.\\
color   & The color of the axis ticks and labels.\\
lstyle  & The line style used for drawing the ticks.\\
ticklen & The length of the ticks.\\
hei     & The height of text used for labelling.\\
off     & Stops GLE from drawing the axis.\\
\end{tabular}

\begin{minipage}[c]{8cm}
\begin{Verbatim}
begin surface
   size 5 5 
   data "surf1.z"
   zaxis min -1 max 3
   base xstep 0.5 ystep 0.5
   back ystep 1 zstep 1 
   right xstep 0.5 zstep 0.5 lstyle 2
   xtitle "X-axis" hei 0.3
   ytitle "Y-axis" hei 0.3
end surface
\end{Verbatim}
\end{minipage}
\hfill
\begin{minipage}[c]{7cm}
\mbox{\includegraphics{utilities/fig/surf2}}
\end{minipage}

\item[{\sf xtitle "{\sf x title}"  [dist {\it v}\ ] [color {\it c}\ ] [hei {\it v}\ ]}]
\index{xtitle}

\item[{\sf ytitle "{\sf y title}"  [dist {\it v}\ ] [color {\it c}\ ] [hei {\it v}\ ]}]
\index{ytitle}

\item[{\sf ztitle "{\sf z title}"  [dist {\it v}\ ] [color {\it c}\ ] [hei {\it v}\ ]}]
\index{ztitle}

\begin{tabular}{ll}
dist 	& Moves the title further away from the axis.\\
color	& Sets the color of the title.\\
hei  	& Sets the hei in cm of the text used for the title.\\
\end{tabular}

\item[{\sf title "{\it main title}"  [dist {\it v}\ ] [color {\it c}\ ] 
[hei {\it v}\ ] }]
\index{title}

\begin{tabular}{ll}	
dist   & Moves the title further away from the axis.\\
color  & Sets the color of the title.\\
hei    & Sets the hei in cm of the text used for the title.\\
\end{tabular}

\item[{\sf rotate\  $\theta$\  $\phi$\  x }]
\index{rotate}
\begin{verbatim}
rotate 10 20 30
\end{verbatim}
Imagine the unit cube is sitting on the front of your
terminal screen, x along the bottom, y up the left hand
side, and z coming towards you.

The first number (10) rotates the cube along the xaxis, ie
hold the right hand side of the cube and rotate your hand
clockwise 10 degrees.

The second number (20) rotates the cube along the yaxis, ie
hold the top of the cube and rotate it 20 degrees clockwise.

The third number is currently ignored.

The default setting is 60 50 0.

\item[{\sf view x y p }]
\index{view}
Sets the perspective, this is where the cube gets smaller
as the lines dissappear towards infinity.

x and y are the position of infinity on your screen.
p is the degree of perspective, 0 = no perspective
and with 1 the back edge of the box will be touching
infinitiy.  Good values are between 0 and 0.6

\begin{minipage}[c]{8cm}
\begin{Verbatim}
begin surface
   size 5 5
   data "surf1.z"
   zaxis min -1
   rotate 85 85 0
   view 0 5 0.7
end surface
\end{Verbatim}
\end{minipage}
\hfill
\begin{minipage}[c]{7cm}
\mbox{\includegraphics{utilities/fig/surf4}}
\end{minipage}

\item[{\sf top [off]\ [lstyle {\it n} \ ]\ [color {\it c}\ ] }]
\index{top}
Sets the features of the top of the surface.
By default the top is on.

(see also {\sf UNDERNEATH, XLINES, YLINES})

\item[{\sf underneath [off]\ [lstyle {\it n} \ ]\ [color {\it c}\ ] }]
\index{underneath}
Sets the features of the under side of the surface.
By default the underneath is off.

(see also TOP, XLINES, YLINES)

\item[{\sf back [zstep {\it v}\ ] [ystep {\it v}\ ] [lstyle {\it l}\ ] [color {\it c}\ ] [nohidden] }]
\index{back}
Draws a grid on the back face of the cube.

By default hidden lines are removed but {\sf NOHIDDEN} will
stop this from happenning.

\item[{\sf base [xstep {\it v}\ ] [ystep {\it v}\ ] [lstyle {\it l}\ ] [color {\it c}\ ] [nohidden] }]
\index{base}
Draws a grid on the base of the cube.

By default hidden lines are removed but {\sf NOHIDDEN} will
stop this from happenning.

\item[{\sf right [zstep {\it v}\ ] [xstep {\it v}\ ] [lstyle {\it l}\ ] [color {\it c}\ ] [nohidden] }]
\index{right}
Draws a grid on the right face of the cube.

By default hidden lines are removed but {\sf NOHIDDEN} will
stop this from happenning.

\item[{\sf skirt on}]
\index{skirt}
Draws a skirt from the edge of the surface to {\sf ZMIN}.

\begin{minipage}[c]{8cm}
\begin{Verbatim}
begin surface
   size 5 5 
   data "surf1.z"
   zaxis min -1 max 3
   xtitle "X-axis"
   ytitle "Y-axis"
   ztitle "Z-axis"
   points "surf3.dat"
   riselines lstyle 2
   marker fcircle
   skirt on
   rotate 60 35 0
   view 2.5 3 0.6
end surface
\end{Verbatim}
\end{minipage}
\hfill
\begin{minipage}[c]{7cm}
\mbox{\includegraphics{utilities/fig/surf3}}
\end{minipage}

\item[{\sf points {\it myfile.dat} }]
\index{points}
Reads in a data file which must have 3 columns (x,y,z)

This is then used for plotting markers and
rise and drop lines.

\item[{\sf marker {\it circle}\  [hei {\it v} \ ] [color {\it c} \ ]}]
\index{marker}
Draws markers at the co-ordinates read from the {\sf POINTS} file.

\item[{\sf droplines [color {\it c}\ ] [lstyle{\it n}\ ] }]
\index{droplines}
Draws lines from the co-ordinates read from the {\sf POINTS} file
down to zmin.

\item[{\sf riselines [color {\it c}\ ] [lstyle {\it n}\ ]}]
\index{riselines}
Draws lines from the co-ordinates read from the {\sf POINTS} file
up to zmax.

\item[{\sf zclip  [min {\it v1}\ ]  [max {\it v2}\ ] }]
\index{zclip}
{\sf ZCLIP} goes through the Z array and sets any Z value smaller
than {\sf MIN} to {\it v1} and sets any value greater 
than {\sf MAX} to {\it v2}.

\end{commanddescription}
